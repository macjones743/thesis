% Appendix B

\chapter{Implementation of the IR Sensor} % Main appendix title

\label{Appendix 2} % For referencing this appendix elsewhere, use \ref{AppendixA}

To achieve the sensitivity required for the sensor, the operating temperature of the IR detector is about 77K, and the detector must be cooled with liquid nitrogen. Using the funnel to avoid spillage of the liquid nitrogen onto the cable connections or viewing window of the sensor, a few hundred milliliters were poured into the hole at the top of the sensor. When the sensor reaches the correct temperature, an eruption of cool gas occurs. It is important to wait until this eruption completes because the buildup of gas can cause the cap to blow off. Once the eruption subsides, the cap can be replaced to the top of the sensor and power can be supplied to the amplifier. 

The amplifier can be utilized in two settings, AC or DC coupled. AC coupling, uses an extra capacitor to filter the DC component out of a signal containing both AC and DC elements. DC coupling allows for both the AC and DC elements to pass. The main consideration for the IR sensor, though is the gain present at each of the coupling settings. AC coupling introduces a 10x gain to the signal, which is useful, as the level of the emission signal is relatively low, at less than 0.5V. Although DC coupling could be utilized, the gain on the amplifier provides a stronger signal to analyze. For typical IR emission tests, be sure the amplifier is set to AC coupled. If the IR sensor is to be run in absorption mode with the IR light source, DC coupling should be used, as the light source provides a relatively large base signal. To switch between AC and DC coupling, simply plug the cable that goes between the amplifier and the IR sensor into the appropriately labeled spot on the amplifier.

After power is supplied to the sensor through the amplifier, it is important to check that the system is aligned. Since the setup is attached to the support structure and not the tube itself, the recoil of the tube or the moving of the tube to replace diaphragms can alter the alignment of the window to the focusing mirror. To re-align the system, a stainless steel LED tube was constructed. This tube fits within the porthole on the back side of the tube, aligned with the port for the IR sensor. With the LED on and lined up with the inside wall of the far side of the tube, check the light source path. Be sure that the thin slit of red light hits the small sensor area. Adjust the flat mirror as needed to align the slit of red light with the IR sensor pickup area. Once the system is aligned, remove the LED tube and replace the plug in that port. The system at this point is aligned and ready for a test.  

\lhead{Appendix B. \emph{IR Sensor Setup}} % This is for the header on each page - perhaps a shortened title
