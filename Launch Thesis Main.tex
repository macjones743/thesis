%%%%%%%%%%%%%%%%%%%%%%%%%%%%%%%%%%%%%%%%%
% Masters/Doctoral Thesis 
% LaTeX Template
% Version 1.43 (17/5/14)
%
% This template has been downloaded from:
% http://www.LaTeXTemplates.com
%
% Original authors:
% Steven Gunn 
% http://users.ecs.soton.ac.uk/srg/softwaretools/document/templates/
% and
% Sunil Patel
% http://www.sunilpatel.co.uk/thesis-template/
%
% License:
% CC BY-NC-SA 3.0 (http://creativecommons.org/licenses/by-nc-sa/3.0/)
%
% Note:
% Make sure to edit document variables in the Thesis.cls file
%
%%%%%%%%%%%%%%%%%%%%%%%%%%%%%%%%%%%%%%%%%

%----------------------------------------------------------------------------------------
%	PACKAGES AND OTHER DOCUMENT CONFIGURATIONS
%----------------------------------------------------------------------------------------

\documentclass[11pt, oneside]{LaunchThesis} % The default font size and one-sided printing (no margin offsets)

\graphicspath{{Pictures/}} % Specifies the directory where pictures are stored
\usepackage[usenames,dvipsnames]{xcolor}
\definecolor{LafMaroon}{cmyk}{0.1,0.97,0.61,0.48}
\usepackage[square, numbers, comma, sort&compress]{natbib} % Use the natbib reference package - read up on this to edit the reference style; if you want text (e.g. Smith et al., 2012) for the in-text references (instead of numbers), remove 'numbers' 
\hypersetup{urlcolor= LafMaroon, colorlinks=true} % Colors hyperlinks in blue - change to black if annoying
\title{\ttitle} % Defines the thesis title - don't touch this

\begin{document}

\frontmatter % Use roman page numbering style (i, ii, iii, iv...) for the pre-content pages

\setstretch{1.3} % Line spacing of 1.3

% Define the page headers using the FancyHdr package and set up for one-sided printing
\fancyhead{} % Clears all page headers and footers
\rhead{\thepage} % Sets the right side header to show the page number
\lhead{} % Clears the left side page header

\pagestyle{fancy} % Finally, use the "fancy" page style to implement the FancyHdr headers

\newcommand{\HRule}{\rule{\linewidth}{0.5mm}} % New command to make the lines in the title page

% PDF meta-data
\hypersetup{pdftitle={\ttitle}}
\hypersetup{pdfsubject=\subjectname}
\hypersetup{pdfauthor=\authornames}
\hypersetup{pdfkeywords=\keywordnames}

%----------------------------------------------------------------------------------------
%	TITLE PAGE
%----------------------------------------------------------------------------------------

\begin{titlepage}
\begin{center}

\href{http://www.lafayette.edu}{\includegraphics[width = 0.35\textwidth,trim = 0mm 80mm 0mm 80mm, clip]{LAF_logo_w_seal_color.pdf}}\\[0.2cm] 

%\textsc{\LARGE \univname}\\[0.5cm] % University name
\textsc{\Large Senior Honors Thesis}\\[0.2cm] % Thesis type 
\textsc{\Large  \deptname}\\[0.2cm] % Thesis type

\HRule \\[0.2cm] % Horizontal line
{\setstretch{1.6}
{\huge \bfseries \ttitle}\\[-0.2cm]}% Thesis title
\HRule \\[0.6 cm] % Horizontal line
 
\begin{minipage}{0.4\textwidth}
\begin{flushleft} \large
\emph{Author:}\\
\href{http://sites.lafayette.edu/rossmant/student-researchers/}{\authornames} % Author name - remove the \href bracket to remove the link
\end{flushleft}
\end{minipage}
\begin{minipage}{0.4\textwidth}
\begin{flushright} \large
\emph{Advisor:} \\
\href{http://sites.lafayette.edu/rossmant}{\supname} % Supervisor name - remove the \href bracket to remove the link  
\end{flushright}
\end{minipage}\\[.6cm]


\begin{minipage}{0.8\textwidth}
\begin{center}
Thesis Committee:\\
\begin{raggedleft}
\textbf{Dr. Tobias Rossmann}, Chair\\
Mechanical Engineering\\
\textbf{Dr. Daniel Sabatino}\\
Mechanical Engineering\\
\textbf{Dr. James Schaefer}\\
Chemical and Biomolecular Engineering\\
\end{raggedleft}
\end{center}
\end{minipage}\\[1cm]
%\null \vfill
\large \textit{A thesis submitted in fulfillment of the requirements\\ for  \degreename}\\[0.0cm] % University requirement text
\textit{in the}\\[0.2cm]
%\groupname\\\% Research group name and department name
\href{http://sites.lafayette.edu/rossmant/research}{\includegraphics[width = 0.60\textwidth,trim = 0mm 70mm 0mm 70mm, clip]{Launch_Circle_Logo.pdf}}\\
 %\GROUPNAME\\[0.2cm] 
{\large \today}\\[0cm] % Date

 
\vfill
\end{center}

\end{titlepage}

%----------------------------------------------------------------------------------------
%	DECLARATION PAGE
%	Your institution may give you a different text to place here
%----------------------------------------------------------------------------------------

\Declaration{

\addtocontents{toc}{\vspace{0 em}} % Add a gap in the Contents, for aesthetics

I, \authornames, declare that this thesis titled, '\ttitle' and the work presented in it are my own. I confirm that:

\begin{itemize} 
\item[\tiny{$\blacksquare$}] This work was done wholly or mainly while in candidature for honors in Mechanical Engineering at Lafayette College.
\item[\tiny{$\blacksquare$}] Where I have consulted the published work of others, this is always clearly attributed.
\item[\tiny{$\blacksquare$}] Where I have quoted from the work of others, the source is always given. With the exception of such quotations, this thesis is entirely my own work.
\item[\tiny{$\blacksquare$}] I have acknowledged all main sources of help.
\item[\tiny{$\blacksquare$}] Where the thesis is based on work done by myself jointly with others, I have made clear exactly what was done by others and what I have contributed myself.\\
\end{itemize}
 
Signed:\\
\rule[1em]{25em}{0.5pt} % This prints a line for the signature
 
Date:\\
\rule[1em]{25em}{0.5pt} % This prints a line to write the date
}

\clearpage % Start a new page

%----------------------------------------------------------------------------------------
%	QUOTATION PAGE
%----------------------------------------------------------------------------------------

\pagestyle{empty} % No headers or footers for the following pages

\null\vfill % Add some space to move the quote down the page a bit

\textit{``60\% of the time, it works every time."}

\begin{flushright}
Brian Fantana
\end{flushright}

\vfill\vfill\vfill\vfill\vfill\vfill\null % Add some space at the bottom to position the quote just right

\clearpage % Start a new page

%----------------------------------------------------------------------------------------
%	ABSTRACT PAGE
%----------------------------------------------------------------------------------------

\addtotoc{Abstract} % Add the "Abstract" page entry to the Contents

\abstract{\addtocontents{toc}{\vspace{0 em}} % Add a gap in the Contents, for aesthetics

Mixing and combustion are important aspects of high speed, air breathing engines. These engines, similar to commercial jet engines, use the atmospheric air as the oxidizer in combustion reactions. However, as the speed of the aircraft increases, the time which the fuel and air can mix into the proper proportions for combustion decreases. For flight speeds of Mach 7 and higher, this time could be well under 1ms. The need for enhanced mixing techniques that allow the fuel and air to mix into the proper conditions in these engines has sparked much research. One proposed method to enhance mixing is a rectangular cavity downstream of the fuel injectors. This cavity has been shown to be a promising means for mixing and flame-holding. 

Also present in these cavities are strong acoustic waves. As the fuel and air travel over these cavities, pressure waves propagate from the back wall to the front wall. Because of the closed nature of these cavities, these acoustic waves oscillate. These acoustic waves could have the potential to enhance mixing within these cavities further, which could aid in the combustion process as conditions within the engine become less ideal. If the fuel and air proportions are not correct, the combustion in the engine could stop, resulting in a stalled engine. However, with the enhanced mixing of the cavity acoustics, the engine could re-initiate combustion, saving the engine and aircraft. 

Utilizing the newly installed expansion tube facility, tests were run with cavity models at engine conditions similar to the conditions that would be experienced by these supersonic, air breathing engines. This investigation will attempt to understand the mechanics of these acoustic waves as they affect mixing and combustion at non-ideal conditions. 
}

\clearpage % Start a new page

%----------------------------------------------------------------------------------------
%	ACKNOWLEDGEMENTS
%----------------------------------------------------------------------------------------

\setstretch{1.3} % Reset the line-spacing to 1.3 for body text (if it has changed)

\acknowledgements{\addtocontents{toc}{\vspace{0 em}} % Add a gap in the Contents, for aesthetics

There are a few people who I would like to acknowledge. First, I would like to acknowledge my research partner, Ray Sanzi. Thank you, Ray, for helping me learn about various compressible flow topics as well as assist me in running tests. I also want to thank you for all the work you put in for our group presentations. We make a pretty good tag team, if I do say so myself. I do not think I could have done this without you. 

I would also like to thank my research advisor, Professor Rossmann. With your constant support and overall excitement about this project, I was inspired to complete this thesis. I appreciate all the times you forced me do the various presentations throughout the year. I understand now that you were doing it to make me more comfortable giving presentations, and not to torture me.

And to the rest of the LAUNCH team, I wish you good luck in your own projects. Know that I am always an email away if you need any clarification or advice. I hope this thesis is sufficient for anyone wishing to continue this research project in the future.

\vspace{11pt}
Cheers,

Mac
}
\clearpage % Start a new page

%----------------------------------------------------------------------------------------
%	LIST OF CONTENTS/FIGURES/TABLES PAGES
%----------------------------------------------------------------------------------------

\pagestyle{fancy} % The page style headers have been "empty" all this time, now use the "fancy" headers as defined before to bring them back

\lhead{\emph{Contents}} % Set the left side page header to "Contents"
\tableofcontents % Write out the Table of Contents

\lhead{\emph{List of Figures}} % Set the left side page header to "List of Figures"
\listoffigures % Write out the List of Figures

\lhead{\emph{List of Tables}} % Set the left side page header to "List of Tables"
\listoftables % Write out the List of Tables

%----------------------------------------------------------------------------------------
%	ABBREVIATIONS
%----------------------------------------------------------------------------------------

\clearpage % Start a new page

\setstretch{1.5} % Set the line spacing to 1.5, this makes the following tables easier to read

\lhead{\emph{Abbreviations}} % Set the left side page header to "Abbreviations"
\listofsymbols{ll} % Include a list of Abbreviations (a table of two columns)
{
\textbf{FPS} & \textbf{F}rames \textbf{P}er \textbf{S}econd \\
\textbf{L/D} & \textbf{L}ength to \textbf{D}epth \\
\textbf{TTL} & \textbf{T}ransistor \textbf{T}ransistor \textbf{L}ogic
%\textbf{Acronym} & \textbf{W}hat (it) \textbf{S}tands \textbf{F}or \\
}

%----------------------------------------------------------------------------------------
%	PHYSICAL CONSTANTS/OTHER DEFINITIONS
%----------------------------------------------------------------------------------------

\clearpage % Start a new page

\lhead{\emph{Physical Constants}} % Set the left side page header to "Physical Constants"

\listofconstants{lrcl} % Include a list of Physical Constants (a four column table)
{
Empirical Constant & $\alpha$ & $=$ & $0.25$ (for cavities with L/D $>$ 4)\\

Empirical Constant & $k$ & $=$ & $0.57$ (for cavities with L/D $>$ 4)\\


% Constant Name & Symbol & = & Constant Value (with units) \\
}

%----------------------------------------------------------------------------------------
%	SYMBOLS
%----------------------------------------------------------------------------------------

\clearpage % Start a new page

\lhead{\emph{Symbols}} % Set the left side page header to "Symbols"

\listofnomenclature{lll} % Include a list of Symbols (a three column table)
{
$m$ & mode number &  \\
$M_{\infty}$ & Freestream Mach Number & \\
$Re_x$ & Reynolds number & \\
$U_{\infty}$ & Freestream Velocity & $m/s$ \\
% Symbol & Name & Unit \\

& & \\ % Gap to separate the Roman symbols from the Greek

$\gamma_{\infty}$ & Freestream Specific Heat Ratio \\
$\rho$ & Density of test gas & $kg/m^3$\\
$\mu$ & Viscosity of test gas & $kg/m-s$ \\
% Symbol & Name & Unit \\}
}
%----------------------------------------------------------------------------------------
%	DEDICATION
%----------------------------------------------------------------------------------------

%\setstretch{1.3} % Return the line spacing back to 1.3
%
%\pagestyle{empty} % Page style needs to be empty for this page
%
%\dedicatory{For/Dedicated to/To my\ldots} % Dedication text
%
%\addtocontents{toc}{\vspace{1em}} % Add a gap in the Contents, for aesthetics

%----------------------------------------------------------------------------------------
%	THESIS CONTENT - CHAPTERS
%----------------------------------------------------------------------------------------
\setstretch{2} 
\mainmatter % Begin numeric (1,2,3...) page numbering

\pagestyle{fancy} % Return the page headers back to the "fancy" style

% Include the chapters of the thesis as separate files from the Chapters folder
% Uncomment the lines as you write the chapters

%\input{Chapters/Chapter1}
% Chapter Template

\chapter{Introduction} % Main chapter title

\label{Chapter2} % Change X to a consecutive number; for referencing this chapter elsewhere, use \ref{ChapterX}

\lhead{Chapter 1. \emph{Introduction}} % Change X to a consecutive number; this is for the header on each page - perhaps a shortened title
\setstretch{2}
%----------------------------------------------------------------------------------------
%	SECTION 1
%----------------------------------------------------------------------------------------

\section{Motivation}

Supersonic combusting ramjet, or scramjet, engines are on the forefront of supersonic transportation development because of their simplicity and promising outlook for steady and reliable supersonic combustion. These engines differ from typical subsonic jet engines, as scramjets have no moving parts and simply rely on shock waves produced at these supersonic speeds to compress the intake air and provide the means for ignition. Shown in Figure \ref{fig:scramjet} is a diagram of a typical scramjet engine.

One challenge facing the production of these engines is producing steady combustion. Much like keeping a match lit in a hurricane, keeping a flame stabilized at supersonic speeds is quite difficult. One proposed method of flame stabilization is a rectangular cavity. These cavities, are able provide a re-circulation zone with high temperatures and combustion radicals for strong combustion to occur. Many experimental studies have tested the flame-holding abilities of cavities in strong combustion cases \cite{ben2000experimental,ben2001cavity,do2009plasma,yilmaz2013investigation}. Strong combustion occurs when the fuel-air mixture is optimized for efficient fuel burning. The mixture in strong combustion cases occur in proper stoichiometric proportions. These studies focused on cavity dimensions and how the length to depth ratio, L/D, affects key ignition and flame holding characteristics, such as stagnation pressure, stagnation temperature, fuel air mixture, and residence time. Ben-Yakar concluded that with L/D ratios between 4 and 10, strong combustion can be sustained in these cavities for total enthalpy flight conditions of Mach 8, 10, and 13\cite{ben2001cavity}. Also noted in several investigations is the presence of strong acoustic waves\cite{unalmis2004cavity,heller1996letter,williams2007supersonic, mcgregor1970drag,luo2011drag, sato1999advanced}. 

Similar to blowing air over an empty bottle to create a tone, the freestream air traveling over and interacting with these cavities produces acoustic waves. In these cavities, a shear layer develops between the high speed freestream and the slower, re-circulating air in the cavity. As the shear layer travels downstream, it begins to drop. This drop in the shear layer is caused by \textbf{EXPLANATION}. By the time the shear layer reaches the downstream wall of the cavity, it has lowered to a point where the interaction of the shear layer with the downstream wall of the cavity produces strong pressure waves, which propagate upstream, ultimately resonating within the cavity. 

Other experimental studies have investigated the acoustic properties of these cavities \cite{unalmis2004cavity,heller1996letter,williams2007supersonic, mcgregor1970drag,luo2011drag, sato1999advanced}. These investigations concluded that the acoustic waves generated by supersonic cavities produce several undesirable effects. One effect, investigated by McGregor is the induced drag associated with rectangular cavities. The effect of pressure waves within these cavities can increase the drag by as much as 250\% \cite{mcgregor1970drag}. These acoustic waves can also have an adverse effect on equipment and the crew. At low frequencies, the resonating acoustic waves can cause structural damage to the engine. At high frequencies, these waves can cause uneasiness in crew members \cite{mcgregor1970drag}.

Conclusions drawn from these investigations have led to the desire to suppress these acoustic waves. Suppressing these waves would reduce drag on the engine and cause less damage to the engine or the crew. However, stabilizing these acoustic waves could reduce the effectiveness of these cavities because mass transfer and residence time are important to flame holding \cite{ben2001cavity}. These acoustic waves could also have the potential to assist in combustion when conditions for weak combustion are present. Weak combustion, as is studied in this investigation, is at lean fuel air mixture conditions. Sato et al.\cite{sato1999advanced} investigated the enhancement of mixing due to acoustic waves. They concluded that mixing was enhanced by these acoustic waves and the rate of enhancement was controlled by the cavity's shape. However, the investigations performed by Sato et al. did not include the cavity as a flame-holder. The cavity was only used to produce the mixing enhancing acoustic waves. 

One study, though, focused on the ability of cavity-induced acoustic waves to enhance mixing \cite{sato1999advanced}. Their experimental setup, as shown in Figure \ref{fig:sato} shows that the wall-mounted cavities were used exclusively as acoustic wave generators to enhance the mixing of the freestream and the injected gas. Their study showed that the cavity acoustic waves produced had enough energy to enhance mixing. This study will attempt to combine the flame-holding characteristics of these cavities with the acoustic wave assisted mixing. The acoustic waves generated inside the cavity should have nearly the same strength as the waves that escaped the cavity in Sato's study. 

Few, if any, experimental studies have investigated the acoustic properties of these cavities as they assist in mixing and the enhancement of combustion at lean conditions. This investigation can be broken into three parts, which together will provide a clear interpretation of a cavity's suitability as an effective flame holder during weak combustion conditions. 


\section{Background}

%-----------------------------------
%	SUBSECTION 1
%-----------------------------------
\subsection{Acoustic Properties}

The first part of the investigation isolated the acoustic properties of the cavities. The frequency of a cavity can be estimated using an empirical equation derived by Heller and Delfs \cite{heller1996letter}, as shown in Equation \ref{eq:freq}. This equation is derived from Rossiter's semi-empirical formula, shown in Equation \ref{eq:freqOG}, which incorporates the coupling that exists between the acoustic waves and the vortex shedding \cite{rossiter1964wind}. Heller and Delfs modified the equation based on their investigation to account for compressibility effects and higher speed of sound within the cavity \cite{ben2001cavity}. The Heller and Delfs equation is estimated to be able to predict the frequency within the cavity $\pm$10\%\cite{heller1996letter}.

\begin{equation}
f_m = \frac{m-\alpha}{\{M_{\infty}+1/k\}} \cdot \frac{U_\infty}{L}
\label{eq:freqOG}
\end{equation}

\begin{equation}
f_m = \frac{m-\alpha}{\{M_{\infty}/\sqrt{1+[(\gamma_{\infty}-1)/2]M_{\infty}^2}+1/k\}} \cdot \frac{U_\infty}{L}
\label{eq:freq}
\end{equation}

Depending on the mode of the strongest acoustic waves, frequencies of these waves will be expected to be between 16.7kHz and 30 kHz for the first mode. It was observed that as L/D increases, the dominant oscillatory mode also increases \cite{ben2001cavity}. However, these base modes should be present within the cavities modeled, but they may not be the strongest modes. For this part of the investigation, three L/D ratios will be investigated: 5, 7, and 9. From the equation, it can be seen that the frequency generated by the cavity is a function of the freestream test gas conditions and length of the cavity. Keeping the freestream conditions nearly identical while changing the length of the cavity will produce different frequencies within the cavity. Data from these tests will lead to the direct comparison of the acoustic properties of the cavities, such as frequency and relative strength, as a function of cavity frequency. Also investigated at each of the L/D ratios was an angled downstream wall to observe the passive suppression of the normally present acoustic waves.

%-----------------------------------
%	SUBSECTION 2
%-----------------------------------

\subsection{Flame Holding}

The second part of the investigation will be used to isolate the combustion and flame-holding properties of the cavities. As shown by the investigations of Ben-Yakar, L/D ratios between 4 and 10 can sustain strong combustion in these cavities \cite{ben2001cavity}. Because this is a new experiment with a new expansion tube facility, it is important to confirm the flame-holding capabilities of these cavities for the environments that can be achieved. For testing, a strong, stoichiometric, fuel-air mixture will be used and each L/D ratio will be tested, along with its angled downstream wall counterpart. If the cavity model shows promising results of flame-holding, the fuel-air mixture will then be diluted with nitrogen to observe the effects of the acoustic waves as the combustion becomes weaker, which is the third part of this investigation. 

For flame holding to occur, certain conditions must be met within the cavity. One of these conditions is recirculation. Recirculation within the cavity allows the fuel and air to slow down and mix to the proper stoichiometric conditions before igniting. The recirculation of a cavity is characterized by its residence time. The residence time is the amount of time in which the test gas remains within the cavity. For the air-breathing scramjet engines, residence time is particularly important. The fuel and air have less than 1ms to mix and ignite, so any extra time the cavity gives to the mixing and ignition process has the potential to be extremely useful. Computational investigations by Gruber et al. show that there exists one large vortex near the downstream edge of the cavity and a secondary vortex exists near the upstream wall of the cavity \cite{gruber2001fundamental}. This downstream vortex, which interacts with the unstable shear layer, controls the mass transfer between the cavity and the freestream gas. 

One last feature of the cavities is that they provide a hot anchor point for the flame. The initial step of the cavity causes a shear layer to form, which separates the high velocity freestream from the low velocity flow within the cavity. When the freestream gas slows down, the kinetic energy within the flow is converted into mostly thermal energy. This increases the temperature within the cavity, allowing a "hot zone," which is hot enough for auto-ignition of the fuel-air mixture. The hot zone provides a point which the flame can latch onto and stabilize itself to continue the combustion process. 


%-----------------------------------
%	SUBSECTION 3
%-----------------------------------

\subsection{Initiation and Extinction}

The third part of the investigation is used to observe the effect that the acoustic waves have on the flame-holding characteristics of the cavities in a non-stoichiometric fuel-air mixture. For these experiments, both flat and angled wall cavities will be tested at the same conditions. Shown by results of the acoustic investigation, the angled walls do suppress the propagating acoustic waves within the cavity, so any observable differences between the results of the angled versus flat downstream wall will be assumed to be due to the cavity acoustics. Ideally, the results of these tests would be to show that the flat downstream wall, and thus the cavity acoustics, do enhance the mixing of the fuel and air, producing stronger combustion than the angled downstream wall. More details about this investigation are included in the Future Works section of this thesis, Chapter \ref{Chapter5}.


%-------------------------------------------------------
%    FIGURES
%------------------------------------------------------
\newpage

\begin{figure}
\centering
\includegraphics[width=\textwidth]{Figures/scramjet.jpg}
\caption[Scramjet Diagram]{Diagram of a typical scramjet engine, showing shocks produced within the engine. \cite{scramjetFig}}
\label{fig:scramjet}
\end{figure}


\begin{figure}
\centering
\includegraphics[height=3in]{Figures/CavMix.jpg}
\caption[Cavity-Actuated Mixing]{Mixing enhanced by acoustic waves produced by a side wall-mounted cavity \cite{sato1999advanced}}
\label{fig:sato}
\end{figure}

\begin{figure}
\centering
\includegraphics[height=3in]{Figures/CavityDiagram.png}
\caption[Diagram of typical cavity]{Typical hypersonic cavity schematic \cite{lazar2008control}.}
\end{figure}

 
%% Chapter Template

\chapter{Theory} % Main chapter title

\label{Chapter2} % Change X to a consecutive number; for referencing this chapter elsewhere, use \ref{ChapterX}

\lhead{Chapter 2. \emph{Theory}} % Change X to a consecutive number; this is for the header on each page - perhaps a shortened title

%----------------------------------------------------------------------------------------
%	SECTION 1
%----------------------------------------------------------------------------------------

\section{Cavity Flame-holding}




%----------------------------------------------------------------------------------------
%	SECTION 2
%----------------------------------------------------------------------------------------

\section{Cavity Acoustics}





%------------------------------------------------------------------
%    FIGURES
%-----------------------------------------------------------------


\begin{figure}
\centering
\includegraphics[height=3in]{Figures/CavityDiagram.png}
\caption[Diagram of typical cavity]{Typical hypersonic cavity schematic \cite{lazar2008control}.}
\end{figure}
% Chapter Template

\chapter{Experimental Setup} % Main chapter title

\label{Chapter3} % Change X to a consecutive number; for referencing this chapter elsewhere, use \ref{ChapterX}

\lhead{Chapter 3. \emph{Experimental Setup}} % Change X to a consecutive number; this is for the header on each page - perhaps a shortened title

%----------------------------------------------------------------------------------------
%	SECTION 1
%----------------------------------------------------------------------------------------

\section{Expansion Tube}

The expansion tube is an impulse flow device similar to a shock tube. With an expansion tube, a high pressure gas is used to accelerate a volume of lower pressure gas to certain conditions required for supersonic testing. For the case of supersonic cavities, these conditions need to be similar to those found in the combustor of a scramjet engine. \textbf{WRITE THE CONDITIONS}. 

%-----------------------------------
%	SUBSECTION 1
%-----------------------------------
\subsection{Sections}

The expansion tube consists of four sections, which are highlighted in Figure \ref{fig:tubelabeled}. The four sections are: the driver, the double diaphragm, the driven, and the expansion, as listed from upstream to downstream. Initial conditions of the tube set by the operator determines test conditions within test section. These initial conditions include pressure ratios between the sections as well as the gases used in the sections.

The driver section is contained at a high pressure at the start of a test. This section is rated to be filled up to 800 psi. Non-combusting tests performed were run with a driver pressure of 225 psi. With higher pressures, faster test velocities of the test gas can be achieved. 

%-----------------------------------
%	SUBSECTION 2
%-----------------------------------

\subsection{Diaphragms}
To separate the four sections, plastic diaphragms were placed at the boundary between these sections. These diaphragms were used to keep a pressure differential between the driver, double diaphragm, and driven sections. The pressure differential between these sections was a maximum of 120 psi for non-combusting tests. Another diaphragm was used to separate the test gas in the driver and the expansion gas in the expansion section. Depending on the test conditions, this diaphragm was required to withstand a maximum pressure differential of 1 psi. 

Different thickness diaphragms were required, depending on the pressure differential between the sections. All diaphragms used for the driver and double diaphragm sections were cut from polycarbonate sheet. In order to determine the required diaphragm thickness, calculations were performed, utilizing known material properties and \textbf{SOME RELATIONSHIP}. Since the diaphragm is expected to expand to a near half sphere before breaking, a thin wall spherical pressure vessel relationship was used to determine the maximum pressure the plastic could withstand. This relationship, shown in Equation \ref{eq:spherePV}, was utilized to determine a range of thicknesses of the polycarbonate sheets to be used for different pressure conditions. 

\begin{equation}
\sigma_{uts} = \frac{P~r}{2t}
\label{eq:spherePV}
\end{equation}

After initial testing of these diaphragms, it was determined this relationship provided an overestimation of about 185\% for the breaking pressure of a specific thickness of diaphragm. Since the polycarbonate sheets come in certain stock thicknesses, several thicknesses were purchased and testing was performed to determine the breaking pressure of each diaphragm. For this testing, a 1/4" polycarbonate sheet was placed at the upstream end of the double diaphragm and the thinner test sample was placed at the downstream end of the double diaphragm. The double diaphragm was then filled slowly. When the downstream diaphragm broke, the highest pressure reached was recorded. This procedure was repeated for other thicknesses of diaphragms. The results from these burst tests is shown in Table \ref{Table:Burst}. A majority of the non-combusting tests run were at a driver pressure of 225 psi, so 0.045" diaphragms were selected, as they have a higher burst pressure than the pressure differential between the driven and the double diaphragm, but not higher than the differential between the driver and the driven sections. These diaphragms reliably broke for each test.

\begin{table}
\centering
\begin{tabular}{|c|c|c|}
\hline
\hline
Thickness (inches) & Trial 1 Burst Pressure (psi) & Trial 2 Burst Pressure (psi)\\ 
\hline \hline
0.010 & 32 & n/a \\

0.015 & 60 & n/a \\

0.020 & 93 & 92 \\

0.030 & 123 & 125 \\

0.045 & 153 & 163 \\

1/16 & 233 & 274 \\

3/32 & 341 & n/a \\

\hline \hline

\end{tabular}
\caption[Diaphragm Burst Pressures]{Diaphragm burst pressures at various thicknesses of polycarbonate sheets.}
\label{Table:Burst}
\end{table}

Occasionally, after a test was run, it was noticed that the pieces of the diaphragm completely broke off, sending these pieces down the tube. Having these large pieces of diaphragm sent down the tube is unwanted. These large pieces can cause serious damage to the model in the test section, as well as damage to other parts of the tube. During one test, a large piece of diaphragm struck the nose of the blunted cylinder model, causing severe damage to the pressure transducer located at the nose. It was also observed that pieces of diaphragm nicked the observation windows on the test section. These damages needed to be avoided, so one proposed solution to this problem was to score the diaphragms. A short, shallow incision on the outside of the plastic in an "X" pattern would create failure modes which the diaphragm should break along. These failure modes cause the diaphragm to petal, ideally opening as wide as the tube, with the entire diaphragm intact. 

Burst tests were performed on several scored diaphragms of 0.045" thickness.  This scoring was performed by hand with a knife, applying light pressure. The resulting score appeared as deep scratches in an "X" pattern. The results of the tests showed no significant decrease in burst pressure. In fact, all of the tests showed a higher burst pressure for the scored diaphragms than for the not scored ones. This could be due to the scoring allowing the plastic to deform further before bursting. It could also be due to the plastic being from a different batch sheet than the plastic used for earlier burst tests. Regardless of the reason, the results showed that the scored diaphragms could still be used. It was also found that good petalling of the diaphragm occurred, with minimal, if any, loss of diaphragm pieces down the tube. Because of these results, scoring of the diaphragms has become a regular step in the setup of the tube for each test.  



%-----------------------------------
%	SUBSECTION 3
%-----------------------------------

\subsection{•}

%----------------------------------------------------------------------------------------
%	SECTION 2
%----------------------------------------------------------------------------------------

\section{Models}



%-----------------------------------
%	SUBSECTION 1
%-----------------------------------
\subsection{Overview and Design Choices}



%-----------------------------------
%	SUBSECTION 2
%-----------------------------------
\subsection{Modular Design}

Because the acoustic properties are the main focus of this thesis, it was important to design a model in which these acoustic properties could change. Frequency is one main acoustic property that was chosen to be varied with these cavities. Using Heller and Delfs relationship, as stated previously, varying the length of the cavity would give different cavity frequencies \cite{heller1996letter}. However, the L/D is also an important parameter in the flame-holding characteristics of these cavities. Combining these need for various lengths with the need for certain range of L/D resulted in a modular design.

For the modular design, a 1/8-inch deep, 1 5/8-inch long cavity was created, as shown in Figure \ref{fig:cavModel}. The length of the cavity was chosen so that inserts could be attached, decreasing the overall length of the cavity, and achieving the desired L/D. Six inserts were manufactured to create L/Ds of 5, 7, and 9. These inserts are shown relative to the base cavity in Figure \ref{fig:cavInserts}. At each L/D, there was one insert with a flat wall and one insert manufactured with a 30$^\circ$ incline. This angled incline, as shown by Ben-Yakar \citep{ben2001cavity}, has the ability to suppress the acoustic waves. This allowed for the comparison of flame-holding abilities of the cavity with and without strong acoustic waves present at each L/D. This modular design gave a relatively wide spectrum of cavity conditions to test with a relatively easy means of changing these conditions for each test. 

%-----------------------------------
%	SUBSECTION 3
%-----------------------------------
\subsection{Implementation}

%----------------------------------------------------------------------------------------
%	SECTION 3
%----------------------------------------------------------------------------------------

\section{High Speed Pressure Transducers}

Placed along the tube at various locations are piezoelectric pressure transducers. These pressure transducers provide both analog and digital data for each test run. The analog data allows for the observation of shock strength and the digital data allows for the calculation of shock velocity. The initial shock wave produced propagates down the tube. When the shock passes by one of these sensors, it registers as a very sharp increase in pressure. Knowing the time at which these shocks arrive at the various transducer locations, along with the location of the transducers relative to each other allows for the shock speed calculation. 

In order to produce the digital signal required for the LabVIEW timer counter program to calculate the time between signals, a simple analog to digital circuit was designed. The design of the circuit was based off of previous work done by Helen Hutches \ref{Hutchens2015}. This circuit, as shown in Figure \ref{fig:timercircuit}, produces a 5V signal when an analog voltage is higher than a certain threshold. This threshold reference voltage is determined by three potentiometers connected to the circuit. The three potentiometers allow three different reference voltages to be set for up to six pressure transducer signals. When the analog voltage is lower than this reference voltage, the output of the circuit is 0V. This allows the digital data acquisition card to read either a HIGH (5V) signal or a LOW (0V) signal and interpret that as a digital signal. 

This digital signal is also used to trigger the camera to record. The LabVIEW program, once getting a signal from a specified pressure transducer will generate a 5V TTL pulse. This pulse is 

%-------------------------------------------------------------
%	 SUBSECTION 1
%-----------------------------------------------------------

\subsection{Circuit}



%----------------------------------------------------------
%	 SUBSECTION 2
%---------------------------------------------------------

\subsection{Implementation and Troubleshooting}




%---------------------------------------------------------
%    SECTION 4
%-----------------------------------------------------------------------------------

\section{Infrared Sensor}

Test time is one important metric needed for data analysis. After a test is run, the conditions of the test gas need to be known as well as for how long these conditions are experienced by the test gas. One way with which to determine that is with an infrared sensor. 

The sensor used with the expansion tube at Lafayette is a Judson J10D series Indium Antemonide (InSb) sensor. These detectors have photovoltaic sensors that produce a current when exposed to infrared radiation.  

%------------------------
%    SUBSECTION 1
%------------------------

\subsection{Alignment}


%------------------------
%	 SUBSECTION 2
%------------------------

\subsection{Calibration}

%------------------------
%	 SUBSECTION 3
%------------------------

\subsection{Operation}

%------------------------
%	 SUBSECTION 4
%------------------------

To achieve the sensitivity required for the sensor, the operating temperature of the IR detector is about 77K. Because of this, the detector must be cooled with liquid nitrogen. Using the funnel to avoid spillage of the liquid nitrogen onto the cable connections or viewing window of the sensor, a few hundred milliliters were poured into the hole at the top of the sensor. When the sensor reaches the correct temperature, an eruption of cool gas occurs. It is important to wait until this eruption completes because the buildup of gas can cause the cap to blow off. Once the eruption subsides, the cap can be replaced to the top of the sensor and power can be supplied to the amplifier. 


%---------------------------------------------------------
%    SECTION 5
%-----------------------------------------------------------------------------------

\section{Schlieren Imaging}

The images captured for the tests were done so using a high speed camera with a schlieren system. A diagram of the schlieren system at Lafayette can be seen in Figure \ref{fig:schlieren}. The camera used for image capturing is a Phantom Miro m310 camera, capable of taking images up to 100 fps. However, as the recording frame rate increases, the resolution of each image decreases. For testing, a nominal frame rate of 77,000 kHz was used, as this provided a sufficient frame rate without sacrificing too much resolution.

The schlieren effect operates on the principle that light refracts in air due to changes in density. This can be observed firsthand on a hot day. The rippling effect one can see above a road on a hot day is the light refracting due to the different densities of the air, caused by differences in local air temperature. Schlieren imaging takes advantage of this principle by placing a knife edge at the focal point of the system. As light is refracted due to a change in density, this light gets blocked out by the knife edge, showing up as a dark spot in the captured image. Light that is not refracted continues through to the camera unblocked. 

This type of imaging is important to see the various shock waves produced during testing. Since shock waves produce very sharp density gradients, a system that is capable of capturing these density gradients at high speeds is very useful. Typical test times experienced by the non-reacting tests were on the order of about 300 microseconds. With this camera, about 23 images were captured for the test time.  With this many frames, it was possible to time correlate the images as well as extract dominant frequencies within the cavities.

Further information about the system, including more detail on how the system works and how to calibrate and align the system at Lafayette can be found in Ray Sanzi's honors thesis \cite{Sanzi2016}.



%-------------------------------------------------------------------
%    FIGURES
%------------------------------------------------------------
\begin{sidewaysfigure}
\centering
\includegraphics[width=\textwidth]{Figures/TubeLabeled.jpg}
\caption[Annotated Expansion Tube]{Annotated photograph of the expansion tube at Lafayette College}
\label{fig:tubelabeled}
\end{sidewaysfigure}

\begin{figure}
\centering
\includegraphics[height = 3in]{Figures/Cavitylabel.jpg}
\caption[Cavity 3D Model]{3D rendering of cavity used in testing}
\label{fig:cavModel}
\end{figure}

\begin{figure}
\centering
\includegraphics[height = 3in]{Figures/CavityInserts.jpg}
\caption[Cavity Model with Inserts]{3D rendering of cavity with various inserts to alter L/D}
\label{fig:cavInserts}
\end{figure}

\begin{figure}
\centering
\includegraphics[height = 3in]{Figures/IRschematic.png}
\caption[IR setup diagram]{Schematic of the IR setup for the expansion tube.}
\label{fig:IRschematic}
\end{figure}

\begin{figure}
\centering
\includegraphics[height = 3in]{Figures/IRLabeled.jpg}
\caption[Labeled photograph of IR setup]{Photograph of actual IR setup in the lab.}
\label{fig:IRlabel}
\end{figure}

\begin{figure}
\centering
\includegraphics[width=\textwidth]{Figures/Schlieren.png}
\caption[Schlieren Diagram]{Diagram of schlieren system at Lafayette}
\label{fig:schlieren}
\end{figure} 
%% Chapter Template

\chapter{Experimentation} % Main chapter title

\label{Chapter4} % Change X to a consecutive number; for referencing this chapter elsewhere, use \ref{ChapterX}

\lhead{Chapter 4. \emph{Experimentation}} % Change X to a consecutive number; this is for the header on each page - perhaps a shortened title

%----------------------------------------------------------------------------------------
%	SECTION 1
%----------------------------------------------------------------------------------------

\section{Phase 1}


%-----------------------------------
%	SUBSECTION 1
%-----------------------------------
\subsection{Setup and Conditions}



%%-----------------------------------
%%	SUBSECTION 2
%%-----------------------------------
%
%\subsection{Subsection 2}


%----------------------------------------------------------------------------------------
%	SECTION 2
%----------------------------------------------------------------------------------------

\section{Phase 2}


%-----------------------------------
%	SUBSECTION 1
%-----------------------------------
\subsection{Setup and Conditions}



%-----------------------------------------------------------------
%   FIGURES
%-----------------------------------------------------------------

\newpage
 
% Chapter Template

\chapter{Experimental Results} % Main chapter title

\label{Chapter4} % Change X to a consecutive number; for referencing this chapter elsewhere, use \ref{ChapterX}

\lhead{Chapter 4. \emph{Experimental Results}} % Change X to a consecutive number; this is for the header on each page - perhaps a shortened title

%----------------------------------------------------------------------------------------
%	SECTION 1
%----------------------------------------------------------------------------------------

\section{Non-Combusting Tests}

In order to directly compare data between the different L/D ratios tested, it was important to keep the conditions nearly the same for all tests. For all non-combusting tests performed, the driver gas was helium at an initial pressure of 225 psi. The test gas was nitrogen at 0.5 psi and the expansion gas was helium at an initial pressure of 0.25 psi. This produced the freestream conditions as follows: M$_\infty$ = 2.32, U$_\infty$ = 2150 m/s, and T$_\infty$ of 411K. 

%-----------------------------------
%	SUBSECTION 2
%-----------------------------------
\subsection{Schlieren}

For each L/D, schlieren images were captured. Figures \ref{fig:5},\ref{fig:7}, and \ref{fig:9} show the effects the L/D ratio has on mixing abilities. It is clear that all of these cavities exhibit mixing and show signs of strong acoustic signals. The images captured for an L/D ratio of 5, however, show more disturbances, relatively, within the cavity itself. It is difficult to say with the images, though, how much more mixing exists within the cavity. Further sharpening of the images

Figure \ref{fig:5-30} shows the cavity with an angled downstream wall. With this image, it is very clear that there are little or no acoustic waves present. This is strong evidence in support of the angled wall's ability to suppress the oscillation acoustic waves. However, with the acoustic waves not present, it may be possible that the mixing within the cavity suffers as a result. This will be investigated with the combustion tests. 

Also captured by the schlieren imaging system is an indication of residence time within these cavities. Dust particles within the test section were dispersed in the flow and some of these particles were drawn into the cavity. Measuring how long these particles take to make one revolution within the cavity can provide an estimate of residence time. Figure \ref{fig:ResTime} shows one such particle traveling over several images. The particle took 68 frames to make one revolution, corresponding to an estimated residence time of about 850 $\mu$s. This corresponds well with the 1ms magnitude residence time achieved by Ben-Yakar \cite{ben2001cavity}.

%-----------------------------------
%    SUBSECTION 3
%-----------------------------------

\subsection{IR}

For some of these tests, IR emission data was captured to measure test time of the tests. The test times extracted from the IR data are displayed in Table \ref{Table:IRtest}. Of these test times, they all correspond to each other, which implies that conditions were reasonably the same for all tests. The theoretical test time, which was extracted from an X-T diagram, as shown in Figure \ref{fig:XT}, was about 559 $\mu$s for theses tests. However, the compressible flow equations used in generating these X-T diagrams assume the flow to be inviscid. However, due to viscous effects that are present in the tube, the experimental test time will be less than the theoretical test time, which is consistent with the data.

\begin{table}[]
\centering
\caption[Test time calculated with IR signal]{Test time from IR signal. M$_\infty$=2.34, Theoretical test time = 559$\mu$s.}
\label{Table:IRtest}
\begin{tabular}{c|c}

Test Number & Test Time ($\mu$s) \\ \hline
046         & 399                      \\ 
048         & 443                      \\ 
049         & 441                      \\ 
050         & 424                      \\ 
\end{tabular}
\end{table}

Each test produced an IR trace similar to the one shown in Figure\ref{fig:IRtrace}. As can be seen in this trace, the slit width was too wide for the test, so the IR sensor captured some light scattered forward, as indicated by the sloped increase in voltage signal. The signal also includes a lower "shelf" signal. This level is currently thought to be the light produced by burning diaphragm particles traveling down the tube. Regardless, the test gas is indicated by the highest level of the signal. Test time is taken to be the time between points halfway between the lowest and highest value of the test signal. With this method, a consistent measure of test time can be achieved regardless of the amount of forward scattering of light. 


%----------------------------------------------------------------------------------------
%	FIGURES
%----------------------------------------------------------------------------------------
\newpage

\begin{figure}
\centering
\includegraphics[height = 3in]{Figures/5.jpg}
\caption[Schlieren image of cavity. L/D = 5.]{Schlieren image of cavity. L/D = 5.}
\label{fig:5}
\end{figure}

\begin{figure}
\centering
\includegraphics[height = 3in]{Figures/5-30.jpg}
\caption[Schlieren image of cavity with angled downstream wall. L/D = 5.]{Schlieren image of cavity. L/D = 5. Downstream wall angle = 30$^\circ$}
\label{fig:5-30}
\end{figure}

\begin{figure}
\centering
\includegraphics[height = 3in]{Figures/7.jpg}
\caption[Schlieren image of cavity. L/D = 7.]{Schlieren image of cavity. L/D = 7.}
\label{fig:7}
\end{figure}

\begin{figure}
\centering
\includegraphics[height = 3in]{Figures/9.jpg}
\caption[Schlieren image of cavity. L/D = 9.]{Schlieren image of cavity. L/D = 9.}
\label{fig:9}
\end{figure}

\begin{figure}[p!]
\centering
\includegraphics[width = \textwidth]{Figures/cavResTime.jpg}
\caption[Time-correlated Schlieren image of cavity. L/D = 5.]{Time-correlated Schlieren image of cavity. L/D = 5. 9 frames out of 68 are shown. Residence time = 850 $\mu$s.}
\label{fig:9}
\end{figure}
\clearpage

\begin{figure}
\centering
\includegraphics[height = 3in]{Figures/XTDiag.png}
\caption[X-T Diagram of Cavity Tests]{X-T diagram generated from the initial conditions of the cavity tests: P4 = 225psi, P1=0.5psi, P$_e$=0.25psi. The test gas was nitrogen.}
\label{fig:XT}
\end{figure}

\begin{figure}
\centering
\includegraphics[height = 3in]{Figures/Emission.jpg}
\caption[IR Signal for Test 046]{IR Signal for Test 046. P4 = 225psi, P1=0.5psi, P$_e$=0.25psi. The test gas was a 95\% Nitrogen and 5\% CO$_2$ mixture. Test Time = 399 $\mu$s.}
\label{fig:IRTrace}
\end{figure}

 
% Chapter Template

\chapter{Future Work} % Main chapter title

\label{Chapter5} % Change X to a consecutive number; for referencing this chapter elsewhere, use \ref{ChapterX}

\lhead{Chapter 5. \emph{Future Work}} % Change X to a consecutive number; this is for the header on each page - perhaps a shortened title

%----------------------------------------------------------------------------------------
%	SECTION 1
%----------------------------------------------------------------------------------------

\section{Future Testing}

Future tests would involve more of the combustion aspects of this investigation. Investigating the flame holding properties of these cavities will be important to understanding the interaction the acoustic waves with combustion. More tests need to be completed for the strong, stoichiometric mixture as the test gas. Each L/D should be tested, both with flat downstream walls as well as angled downstream walls. 

Currently, there is work being performed within the LAUNCH team to introduce a PLIF imaging system, which will be able to utilize chemiluminescence to clearly image the combustion. Overlaying schlieren images with the PLIF images will give a clearer look at how the cavity is interacting with the flow, and how combustion is initiated and sustained. 



%----------------------------------------------------------------------------------------
%	SECTION 3
%----------------------------------------------------------------------------------------

\section{Suggestions for Other Improvements}

Further testing can be done beyond the weak combustion to further the understanding of this investigation. Much of the data from this study, as related to cavity frequencies, is entirely qualitative. The analysis of this study was done by comparing relative strengths of waves seen in Schlieren imaging. It would be good to have more quantitative data to validate the qualitative data captured by the high speed camera. To do this, a pressure transducer can be mounted within the cavity model.  

Along with capturing quantitative data, testing true engine conditions would be the ultimate goal of these cavity tests. These engines do not have a well-mixed fuel and air mixture coming directly into the cavity itself. Incorporating jets of fuel into the system and having positive data showing enhanced mixing and combustion would show that these cavities could be used effectively in scramjet engines. Further details about both of these improvements are explained in the subsequent sections.

One way in which quantitative data could be used in conjunction with the qualitative data captured by the high speed camera would be to mount a pressure transducer within the cavity. A pressure transducer could have the potential to show the strengths of these pressure oscillations. Also, analyzing the pressure data could lead to accurate measurements of the frequency of the waves within the cavity.

Another approach to determining more accurately the frequency of the waves within the cavity could be by using a microphone in the test section. One research team used a microphone in the test section to capture the frequency of the acoustic waves produced by these cavities \cite{yu1994cavity}. In their tests, they placed the microphone near the cavity to measure frequencies that it produced. Using a Fourier transform, they were able to locate which specific frequencies were dominant and the amplitudes of these frequencies. A microphone within the test section might be a good option for non-invasive frequency measurements. The challenges associated with using a microphone in the test section are protecting the microphone from flying diaphragm particles as well as the high temperatures experienced in the test section. 

%-------------------
%   SUBSECTION 3
%-------------------
\subsection{Fuel Injection into Cavities}

In order to fully understand the mixing effects of these cavities as they apply to real-world engines, it is important to understand that these engines inject the fuel into the freestream air. To investigate these cavities further, the test gas can be changed to an air mixture and the hydrogen fuel can be injected through the model upstream of the cavity. However, with the introduction of the jet-in-crossflow injections, there are other flow characteristics introduced, including bow shocks produced by the injected fuel. Currently, there is research being done by a member of the LAUNCH team on the jet-in-crossflow phenomenon. The combination of research could yield significant results that would directly applicable to the understanding of how these engines could be improved with respect to flame holding within the engine. 




%----------------------------------------------------------------------------------------
%	FIGURES
%----------------------------------------------------------------------------------------

\newpage


 

%----------------------------------------------------------------------------------------
%	THESIS CONTENT - APPENDICES
%----------------------------------------------------------------------------------------

\addtocontents{toc}{\vspace{2em}} % Add a gap in the Contents, for aesthetics

\appendix % Cue to tell LaTeX that the following 'chapters' are Appendices

% Include the appendices of the thesis as separate files from the Appendices folder
% Uncomment the lines as you write the Appendices

%% Appendix A

\chapter{Implementation and Troubleshooting of the Timer Counter Circuit} % Main appendix title

\label{Appendix 1} % For referencing this appendix elsewhere, use \ref{AppendixA}

When using the timer counter box, make sure the box is plugged into the power supply. Once the box has power, turn on the power supply associated with the inverting circuit. Once these are powered, the blue amplifier boxes should be turned on to supply power to the transducers. Set the appropriate gains on these boxes. The appropriate gains are based on what magnitude signal is desired. The LabVIEW program will only read signals between 0 and 5V. The pressure transducers output 0-5V for a pressure range of 0 to 100 psi. However, the pressure these transducers will be exposed to are only between 10 and 20 psi, resulting in a 0.5 to 1V signal. Setting the gain to 5 allows that signal to be amplified to to between 2.5 and 5V. Having a larger signal allows for the comparator circuit to operate more easily. One can set the comparator nominal voltage with confidence, as the larger signal allows for a larger margin of error on the set comparator voltage. This gives more confidence that the signal coming into the comparator circuit will be large enough to trigger a digital signal. 

Once the physical systems are powered and ready, run the two LabVIEW programs for the capturing of the digital signals as well as the analog signals. The analog program filename is ''BasicHighSpeedOscillosope-fixed.vi'' and the digital program filename is ''CountersWorking.vi'' For the analog signal, an input is required for sampling rate and time to take data. The typical sampling rate and time for tests is 1MHz and 10,000 $\mu$s. The digital program asks for a user input of ''milliseconds to wait.'' This is the amount of time the program should wait until sending the 5V TTL pulse to the camera. Typically, this value is set to 0. 

One common problem with this circuit is that no digital signal is sent from the circuit itself. To check for this, use a function generator and an oscilloscope. Hook up the oscilloscope to one of the digital out wires of the box as well as the output of the function generator. Table \ref{Table:pins} shows the number of the digital out wire to its corresponding number. Attach the function generator to the corresponding BNC connector. A T-connector is needed to split the output signal from the function generator. Send a 5V amplitude sine wave from the function generator. The corresponding signal from the box should be a square wave. If there is no square wave present, check that the potentiometer is not too high or too low. The square wave should get wider or narrower depending on the position of the potentiometer. If there is still no signal coming from the box, there may be a burnt out op-amp. If, however, there is no signal coming from any of the outputs, it is unlikely all of the op-amps burned out. In this case, there may be a wiring issue. For this, check continuity between the signal coming in and the signal coming out to make sure voltages at each connection are what they are supposed to be. Some alteration of the circuit may be required in this case. However, that is unlikely, and the problem is likely to be that the potentiometer level was set too high.

\begin{table}[]
\centering
\caption[Digital Pin Out Colors]{Digital Out Pin Colors on the comparator circuit box corresponding to their Analog In numbers}
\label{Table:pins}
\begin{tabular}{l|l}

Analog In & Digital Out \\ \hline
1        & Blue       \\ 
2        & Purple     \\ 
3        & Black       \\ 
4        & Yellow       \\ 
5 		 & White		\\
6 		 & Red	      \\
GND      & Green	\\
Extra    & Orange	\\
\end{tabular}
\end{table}

\lhead{Appendix A. \emph{Timer Counter Setup and Troubleshooting}} % This is for the header on each page - perhaps a shortened title

%% Appendix B

\chapter{Implementation of the IR Sensor} % Main appendix title

\label{Appendix 2} % For referencing this appendix elsewhere, use \ref{AppendixA}

To achieve the sensitivity required for the sensor, the operating temperature of the IR detector is about 77K, and the detector must be cooled with liquid nitrogen. Using the funnel to avoid spillage of the liquid nitrogen onto the cable connections or viewing window of the sensor, a few hundred milliliters were poured into the hole at the top of the sensor. When the sensor reaches the correct temperature, an eruption of cool gas occurs. It is important to wait until this eruption completes because the buildup of gas can cause the cap to blow off. Once the eruption subsides, the cap can be replaced to the top of the sensor and power can be supplied to the amplifier. 

The amplifier can be utilized in two settings, AC or DC coupled. AC coupling, uses an extra capacitor to filter the DC component out of a signal containing both AC and DC elements. DC coupling allows for both the AC and DC elements to pass. The main consideration for the IR sensor, though is the gain present at each of the coupling settings. AC coupling introduces a 10x gain to the signal, which is useful, as the level of the emission signal is relatively low, at less than 0.5V. Although DC coupling could be utilized, the gain on the amplifier provides a stronger signal to analyze. For typical IR emission tests, be sure the amplifier is set to AC coupled. If the IR sensor is to be run in absorption mode with the IR light source, DC coupling should be used, as the light source provides a relatively large base signal. To switch between AC and DC coupling, simply plug the cable that goes between the amplifier and the IR sensor into the appropriately labeled spot on the amplifier.

After power is supplied to the sensor through the amplifier, it is important to check that the system is aligned. Since the setup is attached to the support structure and not the tube itself, the recoil of the tube or the moving of the tube to replace diaphragms can alter the alignment of the window to the focusing mirror. To re-align the system, a stainless steel LED tube was constructed. This tube fits within the porthole on the back side of the tube, aligned with the port for the IR sensor. With the LED on and lined up with the inside wall of the far side of the tube, check the light source path. Be sure that the thin slit of red light hits the small sensor area. Adjust the flat mirror as needed to align the slit of red light with the IR sensor pickup area. Once the system is aligned, remove the LED tube and replace the plug in that port. The system at this point is aligned and ready for a test.  

\lhead{Appendix B. \emph{IR Sensor Setup}} % This is for the header on each page - perhaps a shortened title

%% Appendix B

\chapter{IR Test Information} % Main appendix title

\label{Appendix 3} % For referencing this appendix elsewhere, use \ref{AppendixA}
\begin{table}[h!]
\centering
\caption{Basic pressure, timer counter, IR, and test time data from all successful tests utilizing the IR sensor. The test gas labeled CO$_2$ was a 5\% CO$_2$ mixture in nitrogen. The test gas labeled H$_2$ was a 2H$_2$ + O$_2$ + 8N$_2$ mixture. *Staged filling resulted in two levels of IR signal. The first number represents the time the Hydrogen mixture passed the sensor, while the second represents the CO$_2$ mixture.}

\label{IRinfo}
\begin{tabular*}{\textwidth}{|l|
>{\columncolor[HTML]{EFEFEF}}l l
>{\columncolor[HTML]{EFEFEF}}l l
>{\columncolor[HTML]{EFEFEF}}l l
>{\columncolor[HTML]{EFEFEF}}l l
>{\columncolor[HTML]{EFEFEF}}l |}
\hline
Test No.      & \textbf{041}  & \textbf{046}  & 0\textbf{47}  & \textbf{049}  & \textbf{050}  & \textbf{055}    & \textbf{056}    & \textbf{057}        & \textbf{058}         \\ \hline
Test Gas      & N$_2$   & N$_2$   & N$_2$   & N$_2$   & N$_2$   & CO$_2$ & CO$_2$ & H$_2$ & Staged \\ \hline
P4 (psi)      & 225  & 225  & 225  & 225  & 226  & 225    & 224    & 225        & 223         \\ \hline
P1 (psi)      & 0.5  & 0.51 & 0.5  & 0.5  & 0.75 & 0.51   & 0.5    & 0.75       & 0.76        \\ \hline
Pe (psi)      & 0.5  & 0.25 & 0.25 & 0.25 & 0.25 & 0.26   & 0.26   & 0.28       & 0.25        \\ \hline
T1 (s)        & 189  & 210  & N/A  & N/A  & N/A  & 223.4 & 205.9 & 184.9    & 903.2      \\ \hline
T2 (ms)       & 2.47 & N/A  & N/A  & N/A  & N/A  & 1.73 & 1.67 & 1.7733     & 1.67      \\ \hline
T3 (s)        & 147  & N/A  & N/A  & N/A  & N/A  & 157.7 & 147.2 & 170.4      & 150.5      \\ \hline
T4 (s)        & 148  & 153  & N/A  & N/A  & N/A  & 160.1  & 150.9 & 171.3     & 150.0      \\ \hline
IR Time (s)   & 467  & 399  & 466  & 436  & 424  & 429    & 434    & 561        & 333, 473*  \\ \hline
Test Time (s) & 363  & 551  & 544  & 544  & 659  & 544    & 538    & 676        & N/A         \\ \hline
\end{tabular*}
\end{table}



\lhead{Appendix C. \emph{IR Test Information}} % This is for the header on each page - perhaps a shortened title


\addtocontents{toc}{\vspace{2em}} % Add a gap in the Contents, for aesthetics

\backmatter

%----------------------------------------------------------------------------------------
%	BIBLIOGRAPHY
%----------------------------------------------------------------------------------------

\label{Bibliography}

\lhead{\emph{Bibliography}} % Change the page header to say "Bibliography"

\bibliographystyle{unsrtnat} % Use the "unsrtnat" BibTeX style for formatting the Bibliography

\bibliography{Bibliography} % The references (bibliography) information are stored in the file named "Bibliography.bib"

\end{document}  