% Chapter Template

\chapter{Future Work} % Main chapter title

\label{Chapter5} % Change X to a consecutive number; for referencing this chapter elsewhere, use \ref{ChapterX}

\lhead{Chapter 5. \emph{Future Work}} % Change X to a consecutive number; this is for the header on each page - perhaps a shortened title

%----------------------------------------------------------------------------------------
%	SECTION 1
%----------------------------------------------------------------------------------------

\section{Future Testing}

Future tests would involve more of the combustion aspects of this investigation. Investigating the flame holding properties of these cavities will be important to understanding the interaction the acoustic waves with combustion. More tests need to be completed for the strong, stoichiometric mixture as the test gas. Each L/D should be tested, both with flat downstream walls as well as angled downstream walls. 

Currently, there is work being performed within the LAUNCH team to introduce a PLIF imaging system, which will be able to utilize chemiluminescence to clearly image the combustion. Overlaying schlieren images with the PLIF images will give a clearer look at how the cavity is interacting with the flow, and how combustion is initiated and sustained. 



%----------------------------------------------------------------------------------------
%	SECTION 3
%----------------------------------------------------------------------------------------

\section{Suggestions for Other Improvements}

Further testing can be done beyond the weak combustion to further the understanding of this investigation. Much of the data from this study, as related to cavity frequencies, is entirely qualitative. The analysis of this study was done by comparing relative strengths of waves seen in Schlieren imaging. It would be good to have more quantitative data to validate the qualitative data captured by the high speed camera. To do this, a pressure transducer can be mounted within the cavity model.  

Along with capturing quantitative data, testing true engine conditions would be the ultimate goal of these cavity tests. These engines do not have a well-mixed fuel and air mixture coming directly into the cavity itself. Incorporating jets of fuel into the system and having positive data showing enhanced mixing and combustion would show that these cavities could be used effectively in scramjet engines. Further details about both of these improvements are explained in the subsequent sections.

One way in which quantitative data could be used in conjunction with the qualitative data captured by the high speed camera would be to mount a pressure transducer within the cavity. A pressure transducer could have the potential to show the strengths of these pressure oscillations. Also, analyzing the pressure data could lead to accurate measurements of the frequency of the waves within the cavity.

Another approach to determining more accurately the frequency of the waves within the cavity could be by using a microphone in the test section. One research team used a microphone in the test section to capture the frequency of the acoustic waves produced by these cavities \cite{yu1994cavity}. In their tests, they placed the microphone near the cavity to measure frequencies that it produced. Using a Fourier transform, they were able to locate which specific frequencies were dominant and the amplitudes of these frequencies. A microphone within the test section might be a good option for non-invasive frequency measurements. The challenges associated with using a microphone in the test section are protecting the microphone from flying diaphragm particles as well as the high temperatures experienced in the test section. 

%-------------------
%   SUBSECTION 3
%-------------------
\subsection{Fuel Injection into Cavities}

In order to fully understand the mixing effects of these cavities as they apply to real-world engines, it is important to understand that these engines inject the fuel into the freestream air. To investigate these cavities further, the test gas can be changed to an air mixture and the hydrogen fuel can be injected through the model upstream of the cavity. However, with the introduction of the jet-in-crossflow injections, there are other flow characteristics introduced, including bow shocks produced by the injected fuel. Currently, there is research being done by a member of the LAUNCH team on the jet-in-crossflow phenomenon. The combination of research could yield significant results that would directly applicable to the understanding of how these engines could be improved with respect to flame holding within the engine. 




%----------------------------------------------------------------------------------------
%	FIGURES
%----------------------------------------------------------------------------------------

\newpage


