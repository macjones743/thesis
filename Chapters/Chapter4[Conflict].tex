% Chapter Template

\chapter{Experimental Setup} % Main chapter title

\label{Chapter3} % Change X to a consecutive number; for referencing this chapter elsewhere, use \ref{ChapterX}

\lhead{Chapter 3. \emph{Experimental Setup}} % Change X to a consecutive number; this is for the header on each page - perhaps a shortened title

%----------------------------------------------------------------------------------------
%	SECTION 1
%----------------------------------------------------------------------------------------

\section{Expansion Tube}

The expansion tube is an impulse flow device similar to a shock tube. With an expansion tube, a high pressure gas is used to accelerate a volume of lower pressure gas to certain conditions required for supersonic testing. For the case of supersonic cavities, these conditions need to be similar to those found in the combustor of a scramjet engine. \textbf{WRITE THE CONDITIONS}. 

%-----------------------------------
%	SUBSECTION 1
%-----------------------------------
\subsection{Sections}

The expansion tube consists of four sections, which are highlighted in Figure \ref{fig:tubelabeled}. The four sections are: the driver, the double diaphragm, the driven, and the expansion, as listed from upstream to downstream. Initial conditions of the tube set by the operator determines test conditions within test section. These initial conditions include pressure ratios between the sections as well as the gases used in the sections.

The driver section is contained at a high pressure at the start of a test. This section is rated to be filled up to 800 psi. Non-combusting tests performed were run with a driver pressure of 225 psi. With higher pressures, faster velocities of the test gas can be achieved. 

%-----------------------------------
%	SUBSECTION 2
%-----------------------------------

\subsection{Diaphragms}
To separate the four sections, plastic diaphragms were placed at the boundary between these sections. These diaphragms were used to keep a pressure differential between the driver, double diaphragm, and driven sections. The pressure differential between these sections was a maximum of 120 psi for non-combusting tests. Another diaphragm was used to separate the test gas in the driver and the expansion gas in the expansion section. Depending on the test conditions, this diaphragm was required to withstand a maximum pressure differential of 1 psi. 

Different thickness diaphragms were required, depending on the pressure differential between the sections. All diaphragms used for the driver and double diaphragm sections were cut from polycarbonate sheet. In order to determine the required diaphragm thickness, calculations were performed, utilizing known material properties and \textbf{SOME RELATIONSHIP}. Since the diaphragm is expected to expand to a near half sphere before breaking, a thin wall spherical pressure vessel relationship was used to determine the maximum pressure the plastic could withstand. This relationship, shown in Equation \ref{eq:spherePV} \cite{shigley}, was utilized to determine a range of thicknesses of the polycarbonate sheets to be used for different pressure conditions. 

\begin{equation}
\sigma_{uts} = \frac{P~r}{2t}
\label{eq:spherePV}
\end{equation}

After initial testing of these diaphragms, it was determined this relationship provided an overestimation of about 185\% for the breaking pressure of a specific thickness of diaphragm. Since the polycarbonate sheets come in certain stock thicknesses, several thicknesses were purchased and testing was performed to determine the breaking pressure of each diaphragm. For this testing, a 1/4" polycarbonate sheet was placed at the upstream end of the double diaphragm and the thinner test sample was placed at the downstream end of the double diaphragm. The double diaphragm was then filled slowly. When the downstream diaphragm broke, the highest pressure reached was recorded. This procedure was repeated for other thicknesses of diaphragms. The results from these burst tests is shown in Table \ref{Table:Burst}. A majority of the non-combusting tests run were at a driver pressure of 225 psi, so 0.045" diaphragms were selected, as they have a higher burst pressure than the pressure differential between the driven and the double diaphragm, but not higher than the differential between the driver and the driven sections. These diaphragms reliably broke for each test.

\begin{table}
\centering
\begin{tabular}{|c|c|c|}
\hline
\hline
Thickness (inches) & Trial 1 Burst Pressure (psi) & Trial 2 Burst Pressure (psi)\\ 
\hline \hline
0.010 & 32 & n/a \\

0.015 & 60 & n/a \\

0.020 & 93 & 92 \\

0.030 & 123 & 125 \\

0.045 & 153 & 163 \\

1/16 & 233 & 274 \\

3/32 & 341 & n/a \\

\hline \hline

\end{tabular}
\caption[Diaphragm Burst Pressures]{Diaphragm burst pressures at various thicknesses of polycarbonate sheets.}
\label{Table:Burst}
\end{table}

Occasionally, after a test was run, it was noticed that the pieces of the diaphragm completely broke off, sending these pieces down the tube. Having these large pieces of diaphragm sent down the tube is unwanted. These large pieces can cause serious damage to the model in the test section, as well as damage to other parts of the tube. During one test, a large piece of diaphragm struck the nose of the blunted cylinder model, causing severe damage to the pressure transducer located at the nose. It was also observed that pieces of diaphragm nicked the observation windows on the test section. These damages needed to be avoided, so one proposed solution to this problem was to score the diaphragms. A short, shallow incision on the outside of the plastic in an "X" pattern would create failure modes which the diaphragm should break along. These failure modes cause the diaphragm to petal, ideally opening as wide as the tube, with the entire diaphragm intact. 

Burst tests were performed on several scored diaphragms of 0.045" thickness.  This scoring was performed by hand with a knife, applying light pressure. The resulting score appeared as deep scratches in an "X" pattern. The results of the tests showed no significant decrease in burst pressure. In fact, all of the tests showed a higher burst pressure for the scored diaphragms than for the not scored ones. This could be due to the scoring allowing the plastic to deform further before bursting. It could also be due to the plastic being from a different batch sheet than the plastic used for earlier burst tests. Regardless of the reason, the results showed that the scored diaphragms could still be used. It was also found that good petalling of the diaphragm occurred, with minimal, if any, loss of diaphragm pieces down the tube. Because of these results, scoring of the diaphragms has become a regular step in the setup of the tube for each test.  



%-----------------------------------
%	SUBSECTION 3
%-----------------------------------

\subsection{•}

%----------------------------------------------------------------------------------------
%	SECTION 2
%----------------------------------------------------------------------------------------

\section{Models}



%-----------------------------------
%	SUBSECTION 1
%-----------------------------------
\subsection{Overview and Design Choices}



%-----------------------------------
%	SUBSECTION 2
%-----------------------------------
\subsection{Modular Design}

Because the acoustic properties are the main focus of this thesis, it was important to design a model in which these acoustic properties could change. Frequency is one main acoustic property that was chosen to be varied with these cavities. Using Heller and Delfs relationship, as stated previously, varying the length of the cavity would give different cavity frequencies \cite{heller1996letter}. However, the L/D is also an important parameter in the flame-holding characteristics of these cavities. Combining these need for various lengths with the need for certain range of L/D resulted in a modular design.

For the modular design, a 1/8-inch deep, 1 5/8-inch long cavity was created, as shown in Figure \ref{fig:cavModel}. The length of the cavity was chosen so that inserts could be attached, decreasing the overall length of the cavity, and achieving the desired L/D. Six inserts were manufactured to create L/Ds of 5, 7, and 9. These inserts are shown relative to the base cavity in Figure \ref{fig:cavInserts}. At each L/D, there was one insert with a flat wall and one insert manufactured with a 30$^\circ$ incline. This angled incline, as shown by Ben-Yakar \citep{ben2001cavity}, has the ability to suppress the acoustic waves. This allowed for the comparison of flame-holding abilities of the cavity with and without strong acoustic waves present at each L/D. This modular design gave a relatively wide spectrum of cavity conditions to test with a relatively easy means of changing these conditions for each test. 

%-----------------------------------
%	SUBSECTION 3
%-----------------------------------
\subsection{Implementation}

%----------------------------------------------------------------------------------------
%	SECTION 3
%----------------------------------------------------------------------------------------

\section{High Speed Pressure Transducers}

Sed ullamcorper quam eu nisl interdum at interdum enim egestas. Aliquam placerat justo sed lectus lobortis ut porta nisl porttitor. Vestibulum mi dolor, lacinia molestie gravida at, tempus vitae ligula. Donec eget quam sapien, in viverra eros. Donec pellentesque justo a massa fringilla non vestibulum metus vestibulum. Vestibulum in orci quis felis tempor lacinia. Vivamus ornare ultrices facilisis. Ut hendrerit volutpat vulputate. Morbi condimentum venenatis augue, id porta ipsum vulputate in. Curabitur luctus tempus justo. Vestibulum risus lectus, adipiscing nec condimentum quis, condimentum nec nisl. Aliquam dictum sagittis velit sed iaculis. Morbi tristique augue sit amet nulla pulvinar id facilisis ligula mollis. Nam elit libero, tincidunt ut aliquam at, molestie in quam. Aenean rhoncus vehicula hendrerit.

%---------------------------------------------------------
%    SECTION 4
%-----------------------------------------------------------------------------------

\section{Infrared Sensor}

Test time is one important metric needed for data analysis. After a test is run, the conditions of the test gas need to be known as well as for how long these conditions are experienced by the test gas. One way with which to determine that is with an infrared sensor. 

The sensor used with the expansion tube at Lafayette is a Judson J10D series Indium Antemonide (InSb) sensor. These detectors have photovoltaic sensors that produce a current when exposed to infrared radiation. 

%------------------------
%    SUBSECTION 1
%------------------------

\subsection{Alignment}




%------------------------
%	 SUBSECTION 2
%------------------------

\subsection{Calibration}

%------------------------
%	 SUBSECTION 3
%------------------------

\subsection{Operation}

%------------------------
%	 SUBSECTION 4
%------------------------

To achieve the sensitivity required for the sensor, the operating temperature of the IR detector is about 77K. Because of this, the detector must be cooled with liquid nitrogen. Using the funnel to avoid spillage of the liquid nitrogen onto the cable connections or viewing window of the sensor, a few hundred milliliters were poured into the hole at the top of the sensor. When the sensor reaches the correct temperature, an eruption of cool gas occurs. It is important to wait until this eruption completes because the buildup of gas can cause the cap to blow off. Once the eruption subsides, the cap can be replaced to the top of the sensor and power can be supplied to the amplifier. 

The amplifier has two options with which to supply power to the sensor: AC coupled and DC coupled. The DC coupled option is used when the sensor is to be run in absorption mode. The AC coupled is for emission sensing. With AC coupling, there is a gain of 10 on the signal coming from the sensor, which is important as the emission signals are not very large. These signals, including the gain are on the order of about 150mV, whereas the baseline absorption signal reads out at about 4V. Using the sensor in emission mode allows for easier extraction of the test time, as the baseline signal when the test gas is not passing by the viewing path of the sensor should be zero. 


%---------------------------------------------------------
%    SECTION 5
%-----------------------------------------------------------------------------------

\section{Schlieren Imaging}


%-------------------------------------------------------------------
%    FIGURES
%------------------------------------------------------------
\begin{sidewaysfigure}
\centering
\includegraphics[width=\textwidth]{Figures/TubeLabeled.jpg}
\caption[Annotated Expansion Tube]{Annotated photograph of the expansion tube at Lafayette College}
\label{fig:tubelabeled}
\end{sidewaysfigure}

\begin{figure}
\centering
\includegraphics[height = 3in]{Figures/IRschematic.png}
\caption[IR setup diagram]{Schematic of the IR setup for the expansion tube.}
\label{fig:IRschematic}
\end{figure}

\begin{figure}
\centering
\includegraphics[height = 3in]{Figures/IRLabeled.jpg}
\caption[Labeled photograph of IR setup]{Photograph of actual IR setup in the lab.}
\end{figure}